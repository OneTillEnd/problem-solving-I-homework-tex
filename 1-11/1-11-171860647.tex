%%%%%%%%%%%%%%%%%%%%%%%%%%%%%%%%%%%%%%%%%%%%%%%%%%%%%%%%%%%%
% File: hw.tex 						   %
% Description: LaTeX template for homework.                %
%
% Feel free to modify it (mainly the 'preamble' file).     %
% Contact hfwei@nju.edu.cn (Hengfeng Wei) for suggestions. %
%%%%%%%%%%%%%%%%%%%%%%%%%%%%%%%%%%%%%%%%%%%%%%%%%%%%%%%%%%%%

%%%%%%%%%%%%%%%%%%%%%%%%%%%%%%%%%%%%%%%%%%%%%%%%%%%%%%%%%%%%%%%%%%%%%%
% IMPORTANT NOTE: Compile this file using 'XeLaTeX' (not 'PDFLaTeX') %
%
% If you are using TeXLive 2016 on windows,                          %
% you may need to check the following post:                          %
% https://tex.stackexchange.com/q/325278/23098                       %
%%%%%%%%%%%%%%%%%%%%%%%%%%%%%%%%%%%%%%%%%%%%%%%%%%%%%%%%%%%%%%%%%%%%%%

\documentclass[11pt, a4paper, UTF8]{ctexart}
\input{preamble}  % modify this file if necessary

%%%%%%%%%%%%%%%%%%%%
\title{第十一讲:有限与无限}
\me{廖玺然}{171860647}
\date{\today}     % you can specify a date like ``2017年9月30日''.
%%%%%%%%%%%%%%%%%%%%
\begin{document}
\maketitle
%%%%%%%%%%%%%%%%%%%%
\noplagiarism	% always keep this
%%%%%%%%%%%%%%%%%%%%
\beginthishw	% begin ``this homework (hw)'' part

%%%%%%%%%%
\begin{problem}[UD: 20.4]	% NOTE: use '[]' (instead of '()' or '{}') to provide additional information
  (a) Show that the positive rationals $\mathbb{Q}^{+}$ and the negative 
  rationals $\mathbb{Q}^{-}$ are equivalent.\\
  (b) Show that the even and odd integers are equivalent.
\end{problem}

% The ``remark'' environments (when needed) must be 
% put before the ``solution''/``revision''/``proof'' environments.
%\begin{remark}	% Optional
%  以下解答参考了书籍/网站/讲义 $\ldots$。
%
%  \noindent 以下解答是与 XXX 同学讨论得到的。
%\end{remark}

\begin{solution}
  (a) Define a function $f: \mathbb{Q}^{+} \rightarrow \mathbb{Q}^{-}$ 
  by $f(x) = -x$, prove that it's bijective:\\
  $~~~~~~\forall x_{1}, x_{2} \in \mathbb{Q}^{+}$, if $x_{1} \neq x_{2}$, 
  then $-x_{1} \neq -x_{2}$.\\
  $~~~~~~\forall x_{1} \in \mathbb{Q}^{-}$, $\exists x_{2} \in \mathbb{Q}^{+}$, 
  $x_{1} = -x_{2}$.\\
  $~~~~~~$Thus $f$ is bijective, which means that $\mathbb{Q}^{+} \approx \mathbb{Q}^{-}$.\\
  (b) Let $\mathbb{A}$ and $\mathbb{B}$ be the set of all even integers 
  and all odd integers respectively.\\
  $~~~~~~$Define a function $f: \mathbb{A} \rightarrow \mathbb{B}$ by 
  $f(x) = x + 1$, prove that it's bijective:\\
  $~~~~~~\forall x_{1}, x_{2} \in \mathbb{A}$, if $x_{1} \neq x_{2}$, 
  then $x_{1} + 1 \neq x_{2} + 1$.\\
  $~~~~~~\forall x_{1} \in \mathbb{B}$, $\exists x_{2} \in \mathbb{A}$, 
  $x_{1} = x_{2} + 1$.\\
  $~~~~~~$Thus $f$ is bijective, which means that the even and odd 
  integers are equivalent.
\end{solution}
%%%%%%%%%%

%%%%%%%%%%
\begin{problem}[UD: 20.8]
  Prvoe Theorem 20.6 working with the outline given in the text.
\end{problem}

\begin{solution}
  Proof.\\
  $\forall x \in A$, $H(x) = f(x)$, since $f$ is well-defined and 
  bijective, $H|_{A}$ is well-defined and bijective.\\
  For the same reason we can prove that $H|_{B}$ is well-defined 
  and bijective.\\
  Thus $H$ is well-defined and bijective.\\
  Then we can conclude that $A \cup B \approx C \cup D$.
\end{solution}
%%%%%%%%%%
% \newpage  % continue in a new page
%%%%%%%%%%
\begin{problem}[UD: 20.9]
  (a) Suppose that $A$ and $B$ are nonempty finite sets and $A \cap B = \emptyset$. 
  Show that there exist integers $n$ and $m$ such that $A \approx \{ 1,2,...,n \}$ 
  and $B \approx \{ n + 1,...,n + m \}$.\\
  (b) Prove Corollary 20.8
\end{problem}

% \begin{remark}	
%   Refer to book.
% \end{remark}

\begin{solution}
  (a) Since $A$ is finite, $\exists n \in \mathbb{Z}^{+}$, $A \approx \{ 1,2,...,n \}$.\\
  $~~~~~~$Since $B$ is finite, $\exists m \in \mathbb{Z}^{+}$, $B \approx \{ 1,2,...,m \}$.\\
  $~~~~~~$Define a function $f: \{ 1,2,...,m \} \rightarrow \mathbb{Z}^{+}$ by 
  $f(x) = x + n$, we can prove that $f$ is bijective, thus $ran(f) \approx \{ 1,2,...,m \}$.\\
  $~~~~~~$Since $ran(f) = \{ n + 1,...,n + m \}$, we can conclude that 
  $\exists m \in \mathbb{Z}^{+}$, $B \approx \{ n + 1,...,n + m \}$.\\
  (b) From (a) we know that $\exists n, m \in \mathbb{Z}^{+}$, $A \approx \{ 1,2,...,n \}$ 
  and $B \approx \{ n + 1,...,n + m \}$.\\
  $~~~~~~$Let $C$ and $D$ be $\{ 1,2,...,n \}$ and $\{ n + 1,...,n + m \}$ 
  respectively, then $C \cap D = \emptyset$.\\
  $~~~~~~$From Theorem 20.6 we can conclude that $A \cup B \approx C \cup D$.\\
  $~~~~~~$Since $C \cup D = \{ 1,2,...,n + m \}$, we can conclude that 
  $A \cup B$ is finite.
\end{solution}
%%%%%%%%%%%

%%%%%%%%%%%%%%%
\begin{problem}[UD: 20.10]
  Prove Theorem 20.14 below. We suggest that you start by working Problem 15.14 
  if you have not already done so.
\end{problem}

\begin{solution}
  Since $A \approx C$ and $B \approx D$, we can define two bijective functions $f: A \rightarrow C$ and $g: B \rightarrow D$.\\
  Define a function $H: A \times B \rightarrow C \times D$ by 
  $$H(a, c) = (f(a), g(c)).$$\\
  From Problem 15.14 we know that $H$ is bijective.\\
  Thus $A \times B = C \times D$.
\end{solution}
%%%%%%%%%

%%%%%%%%%%%
\begin{problem}[UD: 21.7]
  Show that $\mathbb{Q}$ is infinite.
\end{problem}

\begin{solution}
  We know that $\mathbb{N} \subset \mathbb{Q}$, and Theorem 21.3 tells 
  us that $\mathbb{N}$ is infinite. Since Corollary 20.11 says that 
  every subset of a finite set is finite, our set $\mathbb{Q}$ must be 
  infinite.
\end{solution}
%%%%%%%%

%%%%%%%%%%%
\begin{problem}[UD: 21.9]
  Let $A$ be a set, and suppose that $B$ is an infinite subset of $A$. 
  Show that $A$ must be infinite.
\end{problem}

\begin{solution}
  From Corollary 20.11 we know that $\forall B \subseteq A$, if 
  $A$ is finite, then $B$ is finite, so we know the following 
  contrapositive is true:\\
  $\forall B \subseteq A$, if $B$ is not finite, then $A$ is not finite.\\
  And from the definition of finite sets we know that if a set is 
  not finite, then it is infinite.\\
  Thus according to the conditions in this problem, we can conclude 
  that $A$ is infinite.
\end{solution}
%%%%%%%%%

%%%%%%%%%
\begin{problem}[UD: 21.10]
  Suppose that $A$ is an infinite set, $B$ is a finite set and $f: A \rightarrow B$ 
  is a function. Show that there exists $b \in B$ such that $f^{-1}(\{ b \})$ 
  is infinite.
\end{problem}

\begin{solution}
  Suppose $\forall b \in B$, $f^{-1}(\{ b \})$ is finite,\\
  From Exercise 20.13 we know that $\bigcup _{b \in B} f^{-1}(\{ b \})$ is finite.\\
  But in fact $\bigcup _{b \in B} f^{-1}(\{ b \}) \approx A$, which means that 
  $\bigcup _{b \in B} f^{-1}(\{ b \})$ is infinite, and this contradicts the deduction 
  above.\\
  So we can conclude that $\exists b \in B$, $f^{-1}(\{ b \})$ is infinite.
\end{solution}
%%%%%%%%%

%%%%%%%%%
\begin{problem}[UD: 21.11]
  Let $X$ be an infinite set, and $A$ and $B$ be finite subsets of $X$. 
  Answer each of the following, giving reasons for your answers:\\
  (a) Is $A \cap B$ finite or infinite?\\
  (b) Is $A \backslash B$ finite or infinite?\\
  (c) Is $X \backslash A$ finite or infinite?\\
  (d) Is $A \cup B$ finite or infinite?\\
  (e) If $f: A \rightarrow X$ is a one-to-one function, is $f(A)$ 
  finite or infinite? 
\end{problem}

\begin{solution}
  (a) Finite, cuz $(A \cap B) \subseteq A$ and $A$ is finite.\\
  (b) Finite, cuz $(A \backslash B) \subseteq A$ and $A$ is finite.\\
  (c) Infinite, cuz if it is finite, then $X = (X \backslash A) \cup A$ 
  is finite, which contradicts the conditions.\\
  (d) Finite, according to Theorem 20.12 (The union of two finite sets 
  is finite).\\
  (e) Finite, since $f$ is a one-to-one function, $f(A) \approx A$, 
  thus $f(A)$ is finite.
\end{solution}
%%%%%%%%%

%%%%%%%%%%
\begin{problem}[UD: 21.16]
  (a) Suppose that $A$ is a finite set and $B \subseteq A$. We showed 
  that $B$ is finite. Show that $|B| \leq |A|$.\\
  (b) Suppose that $A$ is a finite set and $B \subseteq A$. Show that 
  if $B \neq A$, then $|B| < |A|$.\\
  (c) Show that if two finite sets $A$ and $B$ satisfy $B \subseteq A$ 
  and $|A| \leq |B|$, then $A = B$.
\end{problem}

\begin{solution}
  (a) Suppose $A \approx \{ 1,2,...,n \}$, $B \approx \{ 1,2,...,m \}$,\\
  $~~~~~~$Then $|A| = n$, $|B| = m$.\\
  $~~~~~~$If $B = A$, according to Theorem 21.6, $m = n$, namely 
  $|A| = |B|$;\\
  $~~~~~~$If $B \subset A$, then $m \neq n$,\\
  $~~~~~~$Suppose $m > n$, we will have that $\exists b \in B$, 
  $\forall a \in A$, $a \neq b$, which contradicts the condition 
  $B \subset A$,\\
  $~~~~~~$Thus $m < n$, namely $|B| < |A|$;\\
  $~~~~~~$So we can conclude that $|B| \leq |A|$.\\
  (b) We have proved it in (a).\\
  (c) Suppose $A \approx \{ 1,2,...,n \}$, $B \approx \{ 1,2,...,m \}$,\\
  $~~~~~~$Since $|A| \leq |B|$, we have that $n \leq m$,\\
  $~~~~~~$Since $B \subseteq A$, from (a) we know $m \leq n$,\\
  $~~~~~~$Thus $m = n$, namely $|A| = |B|$,\\
  $~~~~~~$Since $B \subseteq A$, namely $\forall b \in B$, $\exists a \in A$, 
  $a = b$,\\
  $~~~~~~$We can conclude that $A = B$.
\end{solution}
%%%%%%%%%

%%%%%%%%%
\begin{problem}[UD: 21.17]
  Suppose that $A$ and $B$ are finite sets and $f: A \rightarrow B$ 
  is one-to-one. Show that $|A| \leq |B|$.
\end{problem}

\begin{solution}
  Denote $A \approx \{ 1,2,...,n \}$, $B \approx \{ 1,2,...,m \}$.\\
  Suppose $|A| > |B|$, namely $n > m$,\\
  If it is possible to define a one-to-one function from $A$ to $B$,\\
  Then it must be possible to define a one-to-one function from 
  $\{ 1,2,...,n \}$ to $\{ 1,2,...,m \}$,\\
  But this contradicts Pigeonhole principle,\\
  Thus we can conclude that $|A| \leq |B|$.
\end{solution}
%%%%%%%%%%%%%%%

%%%%%%%%%%%%%%%%%
\begin{problem}[UD: 21.18]
  Let $A$ and $B$ be sets with $A$ finite. Let $f: A \rightarrow B$. 
  Prove that $|ran(f)| \leq |A|$.
\end{problem}

\begin{solution}
  Denote $A \approx \{ 1,2,...,n \}$, then $|A| = n$.\\
  If $f$ is one-to-one, then $ran(f) = \{ f(x_{i}): x_{i} \in A,~i \in \mathbb{Z}, 1 \leq i \leq n \}$,\\
  Thus $ran(f) \approx \{ 1,2,...,n \}$, namely $|ran(f)| = n$,\\
  Namely $|ran(f)| = |A|$;\\
  If $f$ is not one-to-one, then $\exists x_{1}, x{2} \in A$, 
  $f(x_{1}) = f(x_{2})$,\\
  Thus $ran(f) \approx \{ 1,2,...,m \}~(m < n)$,\\
  Thus $|ran(f)| = m < n = |A|$;\\
  So we can conclude that $|ran(f)| \leq |A|$
\end{solution}
%%%%%%%%%%%%%

%%%%%%%%%%%%%
\begin{problem}[UD: 21.19]
  Let $A$ be a finite set. Show that a function $f: A \rightarrow A$ 
  is one-to-one if and only if it is onto. Is this still true if $A$ 
  is infinite?
\end{problem}

\begin{solution}
  $~(i)~$If $f$ is one-to-one, then $|ran(f)| = |A|$,\\
  $~~~~~~$Since $|cod(f)| = |A|$, $ran(f) \subseteq cod(f)$,\\
  $~~~~~~$We have that $ran(f) = cod(f)$, namely $f$ is onto;\\
  $(ii)~$If $f$ is onto,\\
  $~~~~~~$Suppose $\exists x_{1}, x_{2} \in A$, $f(x_{1}) = f(x_{2})$,\\
  $~~~~~~$Then $|ran(f)| < |A| = |cod(f)|$,\\
  $~~~~~~$Thus $\forall x_{1}, x_{2} \in A$, $f(x_{1}) \neq f(x_{2})$,\\
  $~~~~~~$Namely $f$ is one-to-one;
  $(i)$ and $(ii)$ complete the proof.
\end{solution}
%%%%%%%%%%%%

%%%%%%%%%%%
\begin{problem}[UD: 22.1]
  Give an example, if possible, of each of the following:\\
  (a) a countably infinite collection of pairwise disjoint finite sets 
  whose union is countably infinite;\\
  (b) a countably infinite collection of nonempty sets whose union is 
  finite;\\
  (c) a countably infinite collection of pairwise disjoint nonempty 
  sets whose union is finite.
\end{problem}

\begin{solution}
  (a) $\{ \{ i \}: i \in \mathbb{N} \}.$\\
  (b) $\{ \{ i~mod~3\}: i \in \mathbb{N} \}.$\\
  (c) $\{ \{ i~mod~3\}: i \in \mathbb{N} \}.$
\end{solution}
%%%%%%%%%%

%%%%%%%%%%
\begin{problem}[UD: 22.2]
  Which of the following sets are finite? countably infinite? uncountable? 
  Give reasons for your answers for each of the following:\\
  (a) $\{ 1/n: n \in \mathbb{Z} \backslash \{ 0 \} \}$;\\
  (b) $\mathbb{R} \backslash \mathbb{N}$;\\
  (c) $\{ x \in \mathbb{Z}: |x - 7| < |x| \}$;\\
  (d) $2\mathbb{Z} \times 3\mathbb{Z}$;\\
  (e) the set of all lines with rational slopes;\\
  (f) $\mathbb{Q} \backslash \{ 0 \}$;\\
  (g) $\mathbb{N} \backslash \{ 1,3 \}$.
\end{problem}

\begin{solution}
  (a) Countably infinite, cuz it is infinite and is a subset of 
  $\mathbb{Q}$ (referred to Corollary 22.4 and Theorem 22.11).\\
  (b) Uncountable, cuz $(0, 1)$ is uncountable and $(0, 1) \subset \mathbb{R} \backslash \mathbb{N}$
   (referred to Corollary 22.4 and the proof of Theorem 22.12).\\
  (c) Countably infinite, cuz it is infinite and is a subset of 
  $\mathbb{Q}$ (referred to Corollary 22.4 and Theorem 22.11).\\
  (d) Countably infinite, cuz it is infinite and $2\mathbb{Z}$ and 
  $3\mathbb{Z}$ are both countable (referred to Corollary 22.4, 22.10 
  and Theorem 22.11).\\
  (e) Uncountable, cuz it can be described by $\mathbb{R} \times \mathbb{Q}$.\\
  (f) Countably infinite, cuz it is infinite and is a subset of 
  $\mathbb{Q}$ (referred to Corollary 22.4 and Theorem 22.11).\\
  (g) Countably infinite, cuz it is infinite and is a subset of 
  $\mathbb{N}$ (referred to Theorem 22.2).\\
\end{solution}
%%%%%%%%%%%%%%%%%%

%%%%%%%%%%%%%%%%%%%
\begin{problem}[UD: 22.3]
  Is the set of all infinite sequences of 0's and 1's finite, countably 
  infinite, or uncountable? Guess and then prove, please.
\end{problem}

\begin{solution}
  Countably infinite, cuz\\
  Define set $A_{i} = \{ (\{ a_{n} \}, \{ b_{i} \}): a_{n} = b_{n}~
  if~n \neq i,~a_{n} < b_{n}~if~n = i \}$.\\
  Then the set this problem referred to can be defined as $\{ A_{i}: i \in \mathbb{N}^{+} \}$,\\
  $\forall i \in \mathbb{N}^{+}$, $|A_{i}| = 2^{n - 1}$,\\
  Thus $A_{i} \approx \mathbb{N}$,\\
  Then $\{ A_{i}: i \in \mathbb{N}^{+} \} \approx \mathbb{N}^{2}$,\\
  From Theorem 22.8 we know that $\{ A_{i}: i \in \mathbb{N}^{+} \}$ 
  is countable,\\
  So we can conclude that the set of all infinite sequences of 0's 
  and 1's is countably infinite.
\end{solution}
%%%%%%%%%%%%%%%%

%%%%%%%%%%%%%%%%%
\begin{problem}[UD: 22.6]
  Prove Corollary 22.4.
\end{problem}

\begin{solution}
  Suppose $A$ is countable, $B \subseteq A$.\\
  If $A$ is finite, then $B$ is finite, and it is countable;\\
  If $A$ is countably infinite,\\
  $~~~$If $B$ is finite, then $B$ is countable,\\
  $~~~$If $B$ is infinite, we know that there exists a one-to-one 
  function $f: A \rightarrow \mathbb{N}$, so we define $g: B \rightarrow \mathbb{N}$ 
  by $g(x) = f(x),~x \in B$, then $g$ is also one-to-one, thus $B$ is 
  countable;\\
  From all the cases above, we can conclude that every subset of a 
  countable set is countable.
\end{solution}
%%%%%%%%%%%%%%%%

%%%%%%%%%%%%%%%
\begin{problem}[UD: 22.9]
  There is another way to show that $\mathbb{Q}$ is countable. Turn 
  the outline below into a proof by describing the counting process.
\end{problem}

\begin{solution}
  Always start counting from the first row;\\
  Move towards the lower left direction;\\
  When reaching the elements in the first column, 
  go back to the next element in the first row and continue counting.\\
  Then we can find a one-to-one function from $\mathbb{Q}^{+}$ to $\mathbb{N}$.
\end{solution}
%%%%%%%%%%%%

%%%%%%%%%%%%
\begin{problem}[UD: 23.2]
  (a) In $\mathbb{R}$, find the distance of the number 1 to the number 
  3 in the usual metric and in the discrete metric.
  (b) In $\mathbb{R}^{2}$, find the distance of the point $(1,3)$ to 
  the point $(2,5)$ in the usual metric, the taxicab metric, the max 
  metric, and the discrete metric.
\end{problem}

\begin{solution}
  (a) $d_{u}(1,3) = 2$, $d_{d}(1,3) = 1$.\\
  (b) $d_{u}((1,3),(2,5)) = \sqrt5$, $d_{tc}((1,3),(2,5)) = 3$, 
  $d_{m}((1,3),(2,5)) = 2$, $d_{d}((1,3),(2,5)) = 1$.
\end{solution}
%%%%%%%%%%

%%%%%%%%%%%
\begin{problem}[UD: 23.3]
  (a) Sketch the set $\{ (x,y) \in \mathbb{R}^{2}: d_{u}((x,y),(0,0)) < 1 \}$.\\
  (b) Sketch the set $\{ (x,y) \in \mathbb{R}^{2}: d_{tc}((x,y),(0,0)) < 1 \}$.\\
  (c) Sketch the set $\{ (x,y) \in \mathbb{R}^{2}: d_{m}((x,y),(0,0)) < 1 \}$.\\
  (d) Sketch the set $\{ (x,y) \in \mathbb{R}^{2}: d_{d}((x,y),(0,0)) < 1 \}$.\\
  (e) Sketch the set $\{ (x,y,z) \in \mathbb{R}^{3}: d_{u}((x,y,z),(0,0,0)) < 1 \}$.\\
\end{problem}

\begin{solution}
  Please refer to the attachment for the answers.(The target space is 
  marked with shadow or simply surrounded by boundary)
\end{solution}
%%%%%%%%%%%%

%%%%%%%%%%%%%
\begin{problem}[UD: 23.10]
  Let $X$ be the space of polynomials with real coefficients. Define 
  a function $d$ from $X \times X \rightarrow \mathbb{R}$ by $d(p,q) = |p(0) - q(0)|$.
  Is $d$ a metric? If so, prove it. If not, why not?
\end{problem}

\begin{solution}
  Nope, cuz the definiteness of metric tells that if $d(p,q) = |p(0) - q(0)| = 0$, 
  then $p = q$, but in fact we can easily find a pair of $p$ and $q$, 
  for example, $p(x) = x^{2} + 2x + 1$, $q(x) = x^{2} + 1$, they satisfy 
  $|p(0) - q(0)| = 0$, while $p(x) \neq q(x)$.
\end{solution}


%%%%%%%%%%
%%%%%%%%%%%%%%%%%%%%
%\begincorrection	% begin the ``correction'' part (Optional)

%%%%%%%%%%
%\begin{problem}[题号]
%  题目。
%\end{problem}

%\begin{cause}
%  简述错误原因(可选)。
%\end{cause}

% Or use the ``solution'' environment
%\begin{revision}
%  正确解答。
%\end{revision}
%%%%%%%%%%
%%%%%%%%%%%%%%%%%%%%
%\beginfb	% begin the feedback section (Optional)

%你可以写:
%\begin{itemize}
%  \item 对课程及教师的建议与意见
%  \item 教材中不理解的内容
%  \item 希望深入了解的内容
%  \item 等
%\end{itemize}
%%%%%%%%%%%%%%%%%%%%
\beginattch

\begin{problem}[UD: 23.3]
  
\end{problem}

\begin{attachment}
  \fignocaption{width = 0.50\textwidth}{figs/a}
  \fignocaption{width = 0.50\textwidth}{figs/b}
  \fignocaption{width = 0.50\textwidth}{figs/c}
  \fignocaption{width = 0.50\textwidth}{figs/d}
  \fignocaption{width = 0.50\textwidth}{figs/e}
\end{attachment}

%%%%%%%%%%%%%%%%%%%%%%

\end{document}