%%%%%%%%%%%%%%%%%%%%%%%%%%%%%%%%%%%%%%%%%%%%%%%%%%%%%%%%%%%%
% File: hw.tex 						   %
% Description: LaTeX template for homework.                %
%
% Feel free to modify it (mainly the 'preamble' file).     %
% Contact hfwei@nju.edu.cn (Hengfeng Wei) for suggestions. %
%%%%%%%%%%%%%%%%%%%%%%%%%%%%%%%%%%%%%%%%%%%%%%%%%%%%%%%%%%%%

%%%%%%%%%%%%%%%%%%%%%%%%%%%%%%%%%%%%%%%%%%%%%%%%%%%%%%%%%%%%%%%%%%%%%%
% IMPORTANT NOTE: Compile this file using 'XeLaTeX' (not 'PDFLaTeX') %
%
% If you are using TeXLive 2016 on windows,                          %
% you may need to check the following post:                          %
% https://tex.stackexchange.com/q/325278/23098                       %
%%%%%%%%%%%%%%%%%%%%%%%%%%%%%%%%%%%%%%%%%%%%%%%%%%%%%%%%%%%%%%%%%%%%%%

\documentclass[11pt, a4paper, UTF8]{ctexart}
%%%%%%%%%%%%%%%%%%%%%%%%%%%%%%%%%%%
% File: preamble.tex
%%%%%%%%%%%%%%%%%%%%%%%%%%%%%%%%%%%

\usepackage[top = 1.5cm]{geometry}

% Set fonts commands
\newcommand{\song}{\CJKfamily{song}} 
\newcommand{\hei}{\CJKfamily{hei}} 
\newcommand{\kai}{\CJKfamily{kai}} 
\newcommand{\fs}{\CJKfamily{fs}}

\newcommand{\me}[2]{\author{{\bfseries 姓名:}\underline{#1}\hspace{2em}{\bfseries 学号:}\underline{#2}}}

% Always keep this.
\newcommand{\noplagiarism}{
  \begin{center}
    \fbox{\begin{tabular}{@{}c@{}}
      请独立完成作业,不得抄袭。\\
      若参考了其它资料,请给出引用。\\
      鼓励讨论,但需独立书写解题过程。
    \end{tabular}}
  \end{center}
}

% Each hw consists of three parts:
% (1) this homework
\newcommand{\beginthishw}{\part{作业}}
% (2) corrections (Optional)
\newcommand{\begincorrection}{\part{订正}}
% (3) any feedback (Optional)
\newcommand{\beginfb}{\part{反馈}}
% For attached figures or else
\newcommand{\beginatt}{\part{附件}}

% For math
\usepackage{amsmath}
\usepackage{amsfonts}
\usepackage{amssymb}

% Define theorem-like environments
\usepackage[amsmath, thmmarks]{ntheorem}

% For pictures
\usepackage{graphicx} 

\theoremstyle{break}
\theorembodyfont{\song}
\theoremseparator{}
\newtheorem*{problem}{题目}

\theorempreskip{2.0\topsep}
\theoremheaderfont{\kai\bfseries}
\theoremseparator{:}
% \newtheorem*{remark}{注}
\theorempostwork{\bigskip\hrule}
\newtheorem*{solution}{解答}
\theorempostwork{\bigskip\hrule}
\newtheorem*{revision}{订正}

\theoremstyle{plain}
\newtheorem*{cause}{错因分析}
\newtheorem*{remark}{注}

\theoremstyle{break}
\theorempostwork{\bigskip\hrule}
\theoremsymbol{\ensuremath{\Box}}
\newtheorem*{proof}{证明}

\renewcommand\figurename{图}
\renewcommand\tablename{表}

% For figures
% for fig with caption: #1: width/size; #2: fig file; #3: fig caption
\newcommand{\fig}[3]{
  \begin{figure}[htp]
    \centering
      \includegraphics[#1]{#2}
      \caption{#3}
  \end{figure}
}

% for fig without caption: #1: width/size; #2: fig file
\newcommand{\fignocaption}[2]{
  \begin{figure}[htp]
    \centering
    \includegraphics[#1]{#2}
  \end{figure}
}  % modify this file if necessary

%%%%%%%%%%%%%%%%%%%%
\title{第十三讲:布尔代数}
\me{廖玺然}{171860647}
\date{\today}     % you can specify a date like ``2017年9月30日''.
%%%%%%%%%%%%%%%%%%%%
\begin{document}
\maketitle
%%%%%%%%%%%%%%%%%%%%
\noplagiarism	% always keep this
%%%%%%%%%%%%%%%%%%%%
\beginthishw	% begin ``this homework (hw)'' part

%%%%%%%%%%
\begin{problem}	% NOTE: use '[]' (instead of '()' or '{}') to provide additional information
  证明布尔代数是有界有补分配格,有界有补分配格是布尔代数.
\end{problem}

% The ``remark'' environments (when needed) must be 
% put before the ``solution''/``revision''/``proof'' environments.
\begin{remark}	% Optional
  以下解答参考了网站https://wenku.baidu.com/view/0b1018ee0975f46527d3e18d.html
%  \noindent 以下解答是与 XXX 同学讨论得到的。
\end{remark}

\begin{proof}
  (1) Suppose there's a bounded, complemented, distributive lattice $M$,\\
  $~~~~~~$Let the first element be the zero element, the last element be the 
  unit element,\\
  $~~~~~~$Denote $\wedge$ as $*$, $\vee$ as $+$.\\
  $~~~~~~$Since $M$ has commutative law and distributive law,\\
  $~~~~~~$We have\\
  $~~~~~~(1a)~a + b = b + a~~~~~~~~~~~~~~~~~~~~~~~~~~(1b)~a * b = b * a$,\\
  $~~~~~~(2a)~a + (b * c) = (a + b) * (a + c)~~~~~~(2b)~a * (b + c) = (a * b) + (a * c)$,\\
  $~~~~~~$From bounded identities we have\\
  $~~~~~~(3a)~a + 0 = a~~~~~~~~~~~~~~~~~~~~~~~~~~~~~~~(3b)~a * 1 = a$,\\
  $~~~~~~$From complemented identities we have $\forall a \in M,~\exists b \in M,~a + b = 1,~a * b = 0$,\\
  $~~~~~~$From Theorem 14.9 we know $b$ is unique, namely $b = a'$,\\
  $~~~~~~$Thus $M$ is a Boolean algebra.\\
  (2) Suppose there's a Boolean algebra $B$,\\
  $~~~~~~$We have\\
  $~~~~~~(sub1)~\forall a \in B,~a * 0 = 0 + (a * 0) = (a * a') + (a * 0) = a * (a' + 0) = a * a' = 0$\\
  $~~~~~~(sub2)~\forall a \in B,~a + 1 = 1 * (a + 1) = (a + a') * (a + 1) = a + (a' * 1) = a + a' = 1$\\
  $~~~~~~(sub3)~\forall a,b,c \in B$, if $a + b = a + c$ and $a' + b = a' + c$, then\\
  $~~~~~~~~~~~~~~~(a + b) * (a' + b) = (a + c) * (a' + c)$, then\\
  $~~~~~~~~~~~~~~~(a * a') + b = (a * a') + c$, then\\
  $~~~~~~~~~~~~~~~0 + b = 0 + c$, then\\
  $~~~~~~~~~~~~~~~b = c$,
  $~~~~~~$We know $B$ has commutative law\\
  $~~~~~~(1a)~a * b = b * a~~~~~~~~~~~~~~~~~~~~~~(1b)~a + b = b + a$,\\
  $~~~~~~$We can prove the absorption law\\
  $~~~~~~(3a)~a * (a + b) = (a + 0) * (a + b) = a + (0 * b) = a + 0 = a$\\
  $~~~~~~(3b)~a + (a * b) = (a * 1) + (a * b) = a * (1 + b) = a * 1 = a$,\\
  $~~~~~~$Using $(sub1)~0 * b = 0$ and $(sub2)~1 + b = 1$;\\
  $~~~~~~$We can prove the associative law\\
  $~~~~~~(2a)~$Since $a + (a * (b * c)) = a,~a + ((a * b) * c) = (a + (a * b)) * (a + c) = a * (a + c) = a$,\\
  $~~~~~~~~~~~a' + (a * (b * c)) = (a' + a) * (a' + (b * c)) = 1 * (a' + (b * c)) = a' + (b * c),$\\
  $~~~~~~~~~~~a' + ((a * b) * c) = (a' + (a * b)) * (a' + c) = (a' + b) * (a' + c) = a' + (b * c)$\\
  $~~~~~~$From $(sub3)$ we have $a * (b * c) = (a * b) * c$,\\
  $~~~~~~$For the same reason we have\\
  $~~~~~~(2b)~a + (b + c) = (a + b) + c$,\\
  $~~~~~~$Let the zero element be the first element, the unit element be the 
  last element,\\
  $~~~~~~$Denote $+$ as $\vee$, $*$ as $\wedge$,\\
  $~~~~~~$Then $B$ is a lattice,\\
  $~~~~~~$From the identity laws and $(sub1),(sub2)$ we know $B$ is a bounded lattice,\\
  $~~~~~~$From the distributive laws we know $B$ is a distributive lattice,\\
  $~~~~~~$From the complement laws we know $B$ is a complemented lattice,\\
  $~~~~~~$Thus $B$ is a bounded, complemented, distributive lattice.
\end{proof}
%%%%%%%%%%

%%%%%%%%%%
\begin{problem}
  证明SM定理15.6.
\end{problem}

\begin{remark}
    以下解答参考了网站https://wenku.baidu.com/view/0b1018ee0975f46527d3e18d.html
\end{remark}

\begin{proof}
  Suppose $x, y \in B$ and $A = \{ a_{1},a_{2}\cdots,a_{r},b_{1},b_{2}\cdots,b_{s},c_{1},c_{2}\cdots,c_{t} \}$,\\
  $x = a_{1} + \cdots + a_{r} + b_{1} + \cdots + b_{s},~~y = b_{1} + \cdots + b_{s} + c_{1} + \cdots + c_{t}$,\\
  Then $x + y = a_{1} + \cdots + a_{r} + b_{1} + \cdots + b_{s} + c_{1} + \cdots + c_{t}$,\\
  $~~~~~~~~~~x * y = b_{1} + \cdots + b_{s}$,\\
  Thus $f(x + y) = \{ a_{1},\cdots,a_{r},b_{1},\cdots,b_{s},c_{1},\cdots,c_{t} \}$\\
  $~~~~~~~~~~~~~~~~~~~~= \{ a_{1},\cdots,a_{r},b_{1},\cdots,b_{s} \} \cup \{ b_{1},\cdots,b_{s},c_{1},\cdots,c_{t} \}$\\
  $~~~~~~~~~~~~~~~~~~~~= f(x) \cup f(y)$,\\
  $~~~~~~~~~~f(x * y) = \{ b_{1},\cdots,b_{s} \}$\\
  $~~~~~~~~~~~~~~~~~~~~= \{ a_{1},\cdots,a_{r},b_{1},\cdots,b_{s} \} \cap \{ b_{1},\cdots,b_{s},c_{1},\cdots,c_{t} \}$\\
  $~~~~~~~~~~~~~~~~~~~~= f(x) \cap f(y)$,\\
  Let $y = c_{1} + \cdots + c_{t}$,\\
  Then $x + y = 1$ and $x * y = 0$, namely $y = x'$,\\
  Thus $f(x') = \{ c_{1},\cdots,c_{t} \} = \{ a_{1},\cdots,a_{r},b_{1},\cdots,b_{s} \}^{c} = (f(x))^{c}$,\\
  Since $\forall x \in B$, the representation is unique,\\
  $f$ is one-to-one and onto,\\
  Thus $f$ is an isomorphism.
\end{proof}
%%%%%%%%%%
% \newpage  % continue in a new page
%%%%%%%%%%
\begin{problem}
  证明等势(有穷)的布尔代数均同构
\end{problem}

 \begin{remark}	
   以下解答参考了网站https://wenku.baidu.com/view/0b1018ee0975f46527d3e18d.html
 \end{remark}

\begin{proof}
  Suppose $B_{1}$ and $B_{2}$ are both Boolean algebra, and that $|B_{1}| = |B_{2}|$.\\
  Let $A_{1}$ and $A_{2}$ be the set of atoms of $B_{1}$ and $B_{2}$ respectively.\\
  We can easily find that $|A_{1}| = |A_{2}|$,\\
  From the definition we know there's a bijection $f: A_{1} \rightarrow A_{2}$,\\
  Define $\varphi: \mathcal{P}(A_{1}) \rightarrow \mathcal{P}(A_{2})$ by
  $\varphi(x) = \{ f(t): t \in x \}$,\\
  We can easily find that $\varphi$ is a bijection,\\
  For $f: X \rightarrow Y$, if $A, B \subseteq X$, since $f$ is one-to-one\\
  $f(A \cup B) = f(A) \cup f(B)$\\
  $f(A \cap B) = f(A) \cap f(B)$\\
  Also $f(x') = (f(x))'$\\
  Thus $\varphi$ is an isomorphism,\\
  Since there exists an isomorphism from $B_{1}$ to $\mathcal{P}(A_{1})$ and 
  from $\mathcal{P}(A_{2})$ to $B_{2}$ respectively,\\
  Thus there exists an isomorphism from $B_{1}$ to $B_{2}$,\\
  Namely $B_{1}$ and $B_{2}$ are isomorphic.
\end{proof}
%%%%%%%%%%
%%%%%%%%%%%%%%%%%%%%
%\begincorrection	% begin the ``correction'' part (Optional)

%%%%%%%%%%
%\begin{problem}[题号]
%  题目。
%\end{problem}

%\begin{cause}
%  简述错误原因(可选)。
%\end{cause}

% Or use the ``solution'' environment
%\begin{revision}
%  正确解答。
%\end{revision}
%%%%%%%%%%
%%%%%%%%%%%%%%%%%%%%
%\beginfb	% begin the feedback section (Optional)

%你可以写:
%\begin{itemize}
%  \item 对课程及教师的建议与意见
% \item 教材中不理解的内容
%  \item 希望深入了解的内容
%  \item 等
%\end{itemize}
%%%%%%%%%%%%%%%%%%%%
\end{document}