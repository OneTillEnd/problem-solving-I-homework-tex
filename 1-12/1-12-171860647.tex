%%%%%%%%%%%%%%%%%%%%%%%%%%%%%%%%%%%%%%%%%%%%%%%%%%%%%%%%%%%%
% File: hw.tex 						   %
% Description: LaTeX template for homework.                %
%
% Feel free to modify it (mainly the 'preamble' file).     %
% Contact hfwei@nju.edu.cn (Hengfeng Wei) for suggestions. %
%%%%%%%%%%%%%%%%%%%%%%%%%%%%%%%%%%%%%%%%%%%%%%%%%%%%%%%%%%%%

%%%%%%%%%%%%%%%%%%%%%%%%%%%%%%%%%%%%%%%%%%%%%%%%%%%%%%%%%%%%%%%%%%%%%%
% IMPORTANT NOTE: Compile this file using 'XeLaTeX' (not 'PDFLaTeX') %
%
% If you are using TeXLive 2016 on windows,                          %
% you may need to check the following post:                          %
% https://tex.stackexchange.com/q/325278/23098                       %
%%%%%%%%%%%%%%%%%%%%%%%%%%%%%%%%%%%%%%%%%%%%%%%%%%%%%%%%%%%%%%%%%%%%%%

\documentclass[11pt, a4paper, UTF8]{ctexart}
%%%%%%%%%%%%%%%%%%%%%%%%%%%%%%%%%%%
% File: preamble.tex
%%%%%%%%%%%%%%%%%%%%%%%%%%%%%%%%%%%

\usepackage[top = 1.5cm]{geometry}

% Set fonts commands
\newcommand{\song}{\CJKfamily{song}} 
\newcommand{\hei}{\CJKfamily{hei}} 
\newcommand{\kai}{\CJKfamily{kai}} 
\newcommand{\fs}{\CJKfamily{fs}}

\newcommand{\me}[2]{\author{{\bfseries 姓名:}\underline{#1}\hspace{2em}{\bfseries 学号:}\underline{#2}}}

% Always keep this.
\newcommand{\noplagiarism}{
  \begin{center}
    \fbox{\begin{tabular}{@{}c@{}}
      请独立完成作业,不得抄袭。\\
      若参考了其它资料,请给出引用。\\
      鼓励讨论,但需独立书写解题过程。
    \end{tabular}}
  \end{center}
}

% Each hw consists of three parts:
% (1) this homework
\newcommand{\beginthishw}{\part{作业}}
% (2) corrections (Optional)
\newcommand{\begincorrection}{\part{订正}}
% (3) any feedback (Optional)
\newcommand{\beginfb}{\part{反馈}}
% For attached figures or else
\newcommand{\beginatt}{\part{附件}}

% For math
\usepackage{amsmath}
\usepackage{amsfonts}
\usepackage{amssymb}

% Define theorem-like environments
\usepackage[amsmath, thmmarks]{ntheorem}

% For pictures
\usepackage{graphicx} 

\theoremstyle{break}
\theorembodyfont{\song}
\theoremseparator{}
\newtheorem*{problem}{题目}

\theorempreskip{2.0\topsep}
\theoremheaderfont{\kai\bfseries}
\theoremseparator{:}
% \newtheorem*{remark}{注}
\theorempostwork{\bigskip\hrule}
\newtheorem*{solution}{解答}
\theorempostwork{\bigskip\hrule}
\newtheorem*{revision}{订正}

\theoremstyle{plain}
\newtheorem*{cause}{错因分析}
\newtheorem*{remark}{注}

\theoremstyle{break}
\theorempostwork{\bigskip\hrule}
\theoremsymbol{\ensuremath{\Box}}
\newtheorem*{proof}{证明}

\renewcommand\figurename{图}
\renewcommand\tablename{表}

% For figures
% for fig with caption: #1: width/size; #2: fig file; #3: fig caption
\newcommand{\fig}[3]{
  \begin{figure}[htp]
    \centering
      \includegraphics[#1]{#2}
      \caption{#3}
  \end{figure}
}

% for fig without caption: #1: width/size; #2: fig file
\newcommand{\fignocaption}[2]{
  \begin{figure}[htp]
    \centering
    \includegraphics[#1]{#2}
  \end{figure}
}  % modify this file if necessary

%%%%%%%%%%%%%%%%%%%%
\title{第十二讲:偏序关系和格}
\me{廖玺然}{171860647}
\date{\today}     % you can specify a date like ``2017年9月30日''.
%%%%%%%%%%%%%%%%%%%%
\begin{document}
\maketitle
%%%%%%%%%%%%%%%%%%%%
\noplagiarism	% always keep this
%%%%%%%%%%%%%%%%%%%%
\beginthishw	% begin ``this homework (hw)'' part

%%%%%%%%%%
\begin{problem}[SM: 14.32]	% NOTE: use '[]' (instead of '()' or '{}') to provide additional information
  Let B = {a, b, c, d, e, f } be ordered as in Fig. 14-17(b).\\
(a) Find all minimal and maximal elements of B.\\
(b) Does B have a first or last element?\\
(c) List two and find the number of consistent enumerations of B into the set {1, 2, 3, 4, 5, 6}.
\end{problem}

% The ``remark'' environments (when needed) must be 
% put before the ``solution''/``revision''/``proof'' environments.
%\begin{remark}	% Optional
%  以下解答参考了书籍/网站/讲义 $\ldots$。

%  \noindent 以下解答是与 XXX 同学讨论得到的。
%\end{remark}

\begin{solution}
  (a) minimal: $d$, $f$.       maximal: $a$\\
  (b) No first element.        last element: $a$.\\
  (c) $~(i)~f(d)=1,~f(f)=2,~f(e)=3,~f(b)=4,~f(c)=5,~f(a)=6$\\
  $~~~~~~(ii)~g(d)=2,~g(f)=1,~g(e)=3,~g(b)=4,~g(c)=5,~g(a)=6$\\
  $~~~~~~$Totally 11.
\end{solution}
%%%%%%%%%%

%%%%%%%%%%
\begin{problem}[SM: 14.44]
  Suppose the following are three consistent enumerations of an ordered set A = {a, b, c, d}:\\
$[(a, 1), (b, 2), (c, 3), (d, 4)],
[(a, 1), (b, 3), (c, 2), (d, 4)],
[(a, 1), (b, 4), (c, 2), (d, 3)]$\\
Assuming the Hasse diagram D of A is connected, draw D.
\end{problem}

\begin{solution}
  Please see the attachment for the answer.
\end{solution}
%%%%%%%%%%
% \newpage  % continue in a new page
%%%%%%%%%%
\begin{problem}[SM: 14.46]
  Consider the English alphabet A = {a, b, c, . . . , y, z} with the usual (alphabetical) order. Recall $A^{*}$ consists of all
words in A. Let L consist of the following list of elements in $A^{*}$:\\
$~~~~~~$gone, or, arm, go, an, about, gate, one, at, occur\\
(a) Sort L according to the short-lex order, i.e., first by length and then alphabetically.\\
(b) Sort L alphabetically.
\end{problem}

% \begin{remark}	
%   Refer to book.
% \end{remark}

\begin{solution}
  (a) an, at, go, or, arm, one, gate, gone, about, occur.\\
  (b) about, an, arm, at, gate, go, gone, occur, one, or.
\end{solution}
%%%%%%%%%%

%%%%%%%%%%
\begin{problem}[SM: 14.58]
  Show that the isomorphism relation $A \cong B$ for ordered sets is an 
  equivalence relation, that is:\\
  (a) $A \cong A$ for any ordered set $A$. (b) If $A \cong B$, then $B \cong A$. 
  (c) If $A \cong B$ and $B \cong C$, then $A \cong C$.
\end{problem}

\begin{solution}
  (a) Define $f: A \rightarrow A$ by $f = \{ (a,a): a \in A \}$, then\\
  $~~~~~~~(i)~$If $a \precsim a'$, then $f(a) \precsim f(a')$,\\
  $~~~~~~(ii)~$If $a || a'$, then $f(a) || f(a')$,\\
  $~~~~~~$Thus $A \cong A$;\\
  (b) Define $g: A \rightarrow B$ that preserves the order relation,\\
  $~~~~~~$Since $A \cong B$, $g$ is a bijection, $g^{-1}(g(x)) = x$,\\
  $~~~~~~$And $a \precsim a' \Leftrightarrow g(a) \precsim g(a')$, $a || a' \Leftrightarrow g(a) || g(a')$,\\
  $~~~~~~~(i)~$From $a \precsim a' \Leftrightarrow g(a) \precsim g(a')$ we 
  can conclude that\\
  $~~~~~~~~~~~~\forall b,b' \in B$, if $b \precsim b'$, then $g^{-1}(b) \precsim g^{-1}(b')$,\\
  $~~~~~~(ii)~$From $a || a' \Leftrightarrow g(a) || g(a')$ we can conclude that\\
  $~~~~~~~~~~~~\forall b,b' \in B$, if $b || b'$, then $g^{-1}(b) || g^{-1}(b')$;\\
  $~~~~~~$Thus $g^{-1}$ preserves the order relation,\\
  $~~~~~~$Thus $B \cong A$;\\
  (c) Define $f: A \rightarrow B$ and $g: B \rightarrow C$ that preserve the order relations,\\
  $~~~~~~$Since $A \cong B$, $B \cong C$, $f$ and $g$ are both bijections,\\
  $~~~~~~$Thus $g \circ f$ is a bijection, $(g \circ f)(x) = g(f(x))$,\\
  $~~~~~~~(i)~$From $a \precsim a' \Rightarrow f(a) \precsim f(a')$, 
  $f(a) \precsim f(a') \Rightarrow g(f(a)) \precsim g(f(a'))$ we can conclude that\\
  $~~~~~~~~~~~~\forall a,a' \in A$, if $a \precsim a'$, then $g(f(a)) \precsim g(f(a'))$,\\
  $~~~~~~(ii)~$From $a || a' \Rightarrow f(a) || f(a')$, $f(a) || f(a') \Rightarrow g(f(a)) || g(f(a'))$ 
  we can conclude that\\
  $~~~~~~~~~~~~\forall a,a' \in A$, if $a || a'$, then $g(f(a)) || g(f(a'))$;\\
  $~~~~~~$Thus $g \circ f$ preserves the order relation,\\
  $~~~~~~$Thus $A \cong C$.
\end{solution}
%%%%%%%%%%%%%

%%%%%%%%%%%%%
\begin{problem}[SM: 14.62]
  Suppose A and B are well-ordered isomorphic sets. Show that there is only one similarity mapping f : A → B.
\end{problem}

\begin{solution}
  Since $A$ and $B$ are both well-ordered,\\
  Denote $a_{0}$, $b_{0}$ as the first element of $A$ and $B$ respectively,\\
  Let $a_{i+1}$, $b_{i+1}$ be the cover of $a_{i}$ and $b_{i}$ respectively
  ($i \geq 0$, $i \in \mathbb{Z}$),\\
  Since $A$ and $B$ are isomorphic,\\
  We can easily find a similarity mapping $f = \{ (a_{i},b_{i}): a_{i} \in A,~b_{i} \in B,
  ~i \geq 0,~i \in \mathbb{Z} \}$,\\
  Suppose we can define another similarity mapping $g: A \rightarrow B$,\\
  Then $g(A) \subseteq B$,\\
  By transfinite induction, we have that\\
  $~(i)~g(a_{0}) \in B$;\\
  $(ii)~\forall i \geq 0$, $g(a_{i+k}) > g(a_{i})$ ($k > 0,~k \in \mathbb{Z}$),\\
  $~~~~~~$And $\{ g(a_{i+k}): k > 0,~k \in \mathbb{Z} \} \subseteq B$,\\
  $~~~~~~$Thus $g(a_{i}) \in B$;\\
  Then $g(A) = B$.\\
  Thus $g(a_{0}) = b_{0}$,\\
  Since $g(a_{i+1})$ is the cover of $g(a_{i})$,\\
  We have that $g(a_{i}) = b_{i}$,\\
  Thus $f = g$,\\
  Thus there is only one similarity mapping $f: A \rightarrow B$.
\end{solution}
%%%%%%%%%%%%

%%%%%%%%%%%%
\begin{problem}[SM: 14.66]
  Consider the lattice M in Fig. 14-19(b).\\
(a) Find all join-irreducible elements.\\
(b) Find the atoms.\\
(c) Find complements of a and b, if they exist.\\
(d) Express each x in M as the join of irredundant join-irreducible elements.\\
(e) Is M distributive? Complemented?
\end{problem}

\begin{solution}
  (a) $0,~a,~b,~c,~g$.\\
  (b) $a,~b,~c$.\\
  (c) $a$: $g$;  $b$: none.\\
  (d) $d = a \vee c$, $e = b \vee c$, $f = a \vee c$, $I = a \vee g$.\\
  (e) $~(i)~$Nope, cuz $a \vee (b \wedge c) = a$ and $(a \vee b) \wedge (a \vee c) = d$,\\
  $~~~~~~(ii)~$Nope, cuz $b$ has no complement.
\end{solution}
%%%%%%%%%%%%

%%%%%%%%%%%%
\begin{problem}[SM: 14.70]
  Consider the lattice $D_{60} = \{1, 2, 3, 4, 5, 6, 10, 12, 15, 20, 30, 60\}$, the divisors of 60 ordered by divisibility.\\
(a) Draw the diagram of $D_{60}$.\\
(b) Which elements are join-irreducible and which are atoms?\\
(c) Find complements of 2 and 10, if they exist.\\
(d) Express each number x as the join of a minimum number of irredundant join irreducible elements.
\end{problem}

\begin{solution}
  (a) Please see the attachment for the answer.\\
  (b) join-irreducible: $1,~2,~3,~5,~4$;  atoms: $2,~3,~5$.\\
  (c) $2$: none;  $10$: none.\\
  (d) $6 = 2 \vee 3$, $10 = 2 \vee 5$, $15 = 3 \vee 5$, $12 = 3 \vee 4$, 
  $20 = 4 \vee 5$, $30 = 2 \vee 3 \vee 5$, $60 = 4 \vee 3 \vee 5$.
\end{solution}
%%%%%%%%%%%%%

%%%%%%%%%%%%%
\begin{problem}[SM: 14.75]
  A lattice M is said to be modular if whenever a ≤ c we have the law\\
$~~~~~~~~~~~~~~~~$a ∨ (b ∧ c) = (a ∨ b) ∧ c\\
(a) Prove that every distributive lattice is modular.\\
(b) Verify that the non-distributive lattice in Fig. 14-7(b) is modular; hence the converse of (a) is not true.\\
(c) Show that the nondistributive lattice in Fig. 14-7(a) is non-modular. (In fact, one can prove that every non-modular
lattice contains a sublattice isomorphic to Fig. 14-7(a).)
\end{problem}

\begin{solution}
  (a) If $a \leq c$, then $a \vee c = c$,\\
  $~~~~~~$If $M$ is distributive, then\\
  $~~~~~~a \vee (b \wedge c) = (a \vee b) \wedge (a \vee c)$\\
  $~~~~~~~~~~~~~~~~~~~= (a \vee b) \wedge c$.\\
  (b) $0 \vee (a \wedge b) = 0 = (0 \vee a) \wedge b$, thus $\forall x,y \in \{ a,b,c \}$, 
  $0 \vee (x \wedge y) = (0 \vee x) \wedge y$;\\
  $~~~~~~0 \vee (a \wedge I) = a = (0 \vee a) \wedge I = (0 \vee I) \wedge a$, 
  thus $\forall x \in \{ a,b,c \}$, $0 \vee (x \wedge I) = (0 \vee x) \wedge I = (0 \vee I) \wedge x$;\\
  $~~~~~~a \vee (b \wedge I) = I = (a \vee b) \wedge I$, thus $\forall x,y \in \{ a,b,c \}$, 
  $x \vee (y \wedge I) = (x \vee y) \wedge I$.\\
  $~~~~~~$Thus $\forall a,b,c \in M$ that satisfy $a \leq c$, we have $a \vee (b \wedge c) = (a \vee b) \wedge c$,\\
  $~~~~~~$Namely $M$ is modular.\\
  (c) Here $a \vee (b \wedge c) = a$, $(a \vee b) \wedge c = c$,\\
  $~~~~~~$Thus $M$ is not modular.
\end{solution}
%%%%%%%%%%%%%%%%%%%%
%\begincorrection	% begin the ``correction'' part (Optional)

%%%%%%%%%%
%\begin{problem}[题号]
%  题目。
%\end{problem}

%\begin{cause}
%  简述错误原因(可选)。
%\end{cause}

% Or use the ``solution'' environment
%\begin{revision}
%  正确解答。
%\end{revision}
%%%%%%%%%%
%%%%%%%%%%%%%%%%%%%%
%\beginfb	% begin the feedback section (Optional)

%你可以写:
%\begin{itemize}
%  \item 对课程及教师的建议与意见
%  \item 教材中不理解的内容
%  \item 希望深入了解的内容
%  \item 等
%\end{itemize}
%%%%%%%%%%%%%%%%%%%%
\beginattch

%%%%%%%%%%%%
\begin{problem}[SM: 14.44]
  
\end{problem}

\begin{attachment}
  \fignocaption{width = 0.70\textwidth}{figs/drawing_1}
\end{attachment}
%%%%%%%%%%%%

%%%%%%%%%%%%
\begin{problem}[SM: 14.70]
  
\end{problem}

\begin{attachment}
  \newpage
  \fignocaption{width = 0.70\textwidth}{figs/drawing_2}
\end{attachment}
\end{document}