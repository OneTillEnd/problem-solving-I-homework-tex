%%%%%%%%%%%%%%%%%%%%%%%%%%%%%%%%%%%%%%%%%%%%%%%%%%%%%%%%%%%%
% File: hw.tex 						   %
% Description: LaTeX template for homework.                %
%
% Feel free to modify it (mainly the 'preamble' file).     %
% Contact hfwei@nju.edu.cn (Hengfeng Wei) for suggestions. %
%%%%%%%%%%%%%%%%%%%%%%%%%%%%%%%%%%%%%%%%%%%%%%%%%%%%%%%%%%%%

%%%%%%%%%%%%%%%%%%%%%%%%%%%%%%%%%%%%%%%%%%%%%%%%%%%%%%%%%%%%%%%%%%%%%%
% IMPORTANT NOTE: Compile this file using 'XeLaTeX' (not 'PDFLaTeX') %
%
% If you are using TeXLive 2016 on windows,                          %
% you may need to check the following post:                          %
% https://tex.stackexchange.com/q/325278/23098                       %
%%%%%%%%%%%%%%%%%%%%%%%%%%%%%%%%%%%%%%%%%%%%%%%%%%%%%%%%%%%%%%%%%%%%%%

\documentclass[11pt, a4paper, UTF8]{ctexart}
%%%%%%%%%%%%%%%%%%%%%%%%%%%%%%%%%%%
% File: preamble.tex
%%%%%%%%%%%%%%%%%%%%%%%%%%%%%%%%%%%

\usepackage[top = 1.5cm]{geometry}

% Set fonts commands
\newcommand{\song}{\CJKfamily{song}} 
\newcommand{\hei}{\CJKfamily{hei}} 
\newcommand{\kai}{\CJKfamily{kai}} 
\newcommand{\fs}{\CJKfamily{fs}}

\newcommand{\me}[2]{\author{{\bfseries 姓名:}\underline{#1}\hspace{2em}{\bfseries 学号:}\underline{#2}}}

% Always keep this.
\newcommand{\noplagiarism}{
  \begin{center}
    \fbox{\begin{tabular}{@{}c@{}}
      请独立完成作业,不得抄袭。\\
      若参考了其它资料,请给出引用。\\
      鼓励讨论,但需独立书写解题过程。
    \end{tabular}}
  \end{center}
}

% Each hw consists of three parts:
% (1) this homework
\newcommand{\beginthishw}{\part{作业}}
% (2) corrections (Optional)
\newcommand{\begincorrection}{\part{订正}}
% (3) any feedback (Optional)
\newcommand{\beginfb}{\part{反馈}}
% For attached figures or else
\newcommand{\beginatt}{\part{附件}}

% For math
\usepackage{amsmath}
\usepackage{amsfonts}
\usepackage{amssymb}

% Define theorem-like environments
\usepackage[amsmath, thmmarks]{ntheorem}

% For pictures
\usepackage{graphicx} 

\theoremstyle{break}
\theorembodyfont{\song}
\theoremseparator{}
\newtheorem*{problem}{题目}

\theorempreskip{2.0\topsep}
\theoremheaderfont{\kai\bfseries}
\theoremseparator{:}
% \newtheorem*{remark}{注}
\theorempostwork{\bigskip\hrule}
\newtheorem*{solution}{解答}
\theorempostwork{\bigskip\hrule}
\newtheorem*{revision}{订正}

\theoremstyle{plain}
\newtheorem*{cause}{错因分析}
\newtheorem*{remark}{注}

\theoremstyle{break}
\theorempostwork{\bigskip\hrule}
\theoremsymbol{\ensuremath{\Box}}
\newtheorem*{proof}{证明}

\renewcommand\figurename{图}
\renewcommand\tablename{表}

% For figures
% for fig with caption: #1: width/size; #2: fig file; #3: fig caption
\newcommand{\fig}[3]{
  \begin{figure}[htp]
    \centering
      \includegraphics[#1]{#2}
      \caption{#3}
  \end{figure}
}

% for fig without caption: #1: width/size; #2: fig file
\newcommand{\fignocaption}[2]{
  \begin{figure}[htp]
    \centering
    \includegraphics[#1]{#2}
  \end{figure}
}  % modify this file if necessary

%%%%%%%%%%%%%%%%%%%%
\title{期末订正}
\me{廖玺然}{171860647}{计算机类}
\date{\today}     % you can specify a date like ``2017年9月30日''.
%%%%%%%%%%%%%%%%%%%%
\begin{document}
\maketitle
%%%%%%%%%%%%%%%%%%%%
\noplagiarism	% always keep this
%%%%%%%%%%%%%%%%%%%%

%%%%%%%%%%%%%%%%%%%%
\begincorrection	% begin the ``correction'' part (Optional)

%%%%%%%%%%
\begin{problem}[DH: 2.10]
    A permutation (a1,...,aN) can be represented by a vector P of length N with P[I] = aI. 
    Design an algorithm which, given an integer N and a vector of integers P of length N, 
    checks whether P represents any permutation of AN.
\end{problem}

\begin{cause}
  未考虑P中值的范围。
\end{cause}

% Or use the ``solution'' environment
\begin{revision}
    (1) read N, P[N]; //输入N和P[N]\\
    (2) define integer X, Y; //声明整形变量X, Y\\
    (3) for X going from 1 to N-1 do the following: //P[N]中每两个数之间比较一次\\
    $~~~$     (3.1) for Y going from X+1 to N do the following:\\
    $~~~~~~$         (3.1.1) if P[X] = P[Y] then\\
    $~~~~~~~~~$               (3.1.1.1) output \"P is not a permutation of AN.\";\\
    $~~~~~~~~~$               (3.1.1.2) goto (5);\\
    $~~~~~~$         (3.1.2) if P[X] > N then \\
    $~~~~~~~~~$               (3.1.2.1) output “P is not a permutation of ”;\\
    $~~~~~~~~~$               (3.1.2.2) goto (5)\\
      (4) output \"P is a permutation of AN.\";\\
      (5) end. //算法终止
\end{revision}
%%%%%%%%%%
\begin{problem}[UD: 7.8]
    Consider the following sets:\\
(i)$(A \cap B) \backslash (A \cap B \cap C)$\\
(ii)$A \cap B \backslash (A \cap B \cap C)$\\
(iii)$A \cap B \cap C^{C}$\\
(iv)$(A \cap B) \backslash C$\\
(v)$(A \backslash C) \cap (B \backslash C)$\\
(a) Which of the sets above are written ambiguously, if any?\\
(b) Of the sets above that make sense, which ones equal the set sketched in Figure 7.2?\\
\end{problem}

\begin{revision}
    (a) $(ii)$\\
    (b) all except $(ii)$
\end{revision}
%%%%%%%%%%%%%%%
\begin{problem}[UD: 8.9]
    Guess a simpler way to express the set A defined as\\
    $$A = \mathbb{Q}\backslash\bigcap_{n \in \mathbb{Z}}(\mathbb{R}\backslash\{2n\}),$$
and then prove that your guess is correct.
\end{problem}

\begin{revision}
    $A = \bigcup_{n \in \mathbb{Z}}\{2n\},$\\
    $\forall x \in \bigcup_{n \in \mathbb{Z}}\{2n\},~~
    x \notin \bigcap_{n \in \mathbb{Z}}(\mathbb{R}\backslash\{2n\})$ and 
    $x \in \mathbb{Q},$\\
    Thus $x \in \mathbb{Q}\backslash\bigcap_{n \in \mathbb{Z}}(\mathbb{R}\backslash\{2n\}),$\\
    $\forall x \notin \bigcup_{n \in \mathbb{Z}}\{2n\},~~x \in \bigcap_{n \in \mathbb{Z}}(\mathbb{R}\backslash\{2n\}),$\\
    Thus $x \notin \mathbb{Q}\backslash\bigcap_{n \in \mathbb{Z}}(\mathbb{R}\backslash\{2n\}).$
\end{revision}
%%%%%%%%%%%%%%%
\begin{problem}[UD: 8.11]
    A collection of sets $\{A_{\alpha\}: \alpha \in I}$ is said to be a pairwise disjoint collection if the following is satisfied: 
    For all $\alpha,\beta \in I$, if $A_{\alpha} \cap A_{\beta} \neq \emptyset$, then $A_{\alpha} = A_{\beta}$. 
    Suppose that each set $A_{\alpha}$ is nonempty.\\
    (g) If $\bigcap_{\alpha \in I}A_{\alpha} = \emptyset$, is $\{A_{\alpha}:\alpha \in I\}$ 
    necessarily a pairwise disjoint collection of sets?
\end{problem}

\begin{revision}
    Nope, cuz suppose $A_{\alpha_{1}} \cap A_{\alpha_{2}} \neq \emptyset,~~A_{\alpha_{1}} \neq A_{\alpha_{2}}$,\\
    and suppose $A_{\alpha_{3}} \cap A_{\alpha_{1}} = \emptyset,~~A_{\alpha_{3}} \cap A_{\alpha_{2}} = \emptyset$,\\
    then $\bigcap_{\alpha \in I}A_{\alpha} = \emptyset$ but $\{A_{\alpha}:\alpha \in I\}$ is not a 
    pairwise disjoint collection of sets.
\end{revision}
%%%%%%%%%%%%%%%
\begin{problem}[UD: 11.9]
    Let X be a nonempty set and $\{A_{\alpha}: \alpha \in I\}$ be a partition of X.\\
    (b) Suppose further that $A_{\alpha} \neq X$ for every $\alpha \in I$. 
    Is $\{X \backslash A_{\alpha}: \alpha \in I\}$ a partition of X? Prove it or give a counterexample.
\end{problem}

\begin{revision}
    (b) Nope, let $X = \{1,2,3\},~~A_{\alpha_{1}} = \{1\},~~A_{\alpha_{2}} = \{2\},~~
    A_{\alpha_{3}} = \{3\},$\\
    $~~~~~~$Then $X\backslash A_{\alpha_{1}} = \{2,3\},~~X\backslash A_{\alpha_{2}} = \{1,3\},$\\
    $~~~~~~X\backslash A_{\alpha_{1}}\cap X\backslash A_{\alpha_{2}} \neq \emptyset,$\\
    $~~~~~~$Thus $\{X \backslash A_{\alpha}: \alpha \in I\}$ is not a partition of X.
\end{revision}
%%%%%%%%%%%%%%
\begin{problem}[UD: 14.13]
    Let $F([0,1])$ denote the set of all real-valued functions defined on the closed 
  interval $[0,1]$. Define a new function $\phi:F([0,1]) \rightarrow \mathbb{R}$ by 
  $\phi(f) = f(0)$. Is it onto? Remember to provide examples where appropriate.
\end{problem}

\begin{cause}
    未理解清楚集合所指对象。
\end{cause}

\begin{revision}
    Yep.\\
    Since $F([0,1])$ denotes the set of all real-valued functions defined on the closed 
    interval $[0,1]$,\\
    $\forall x \in \mathbb{R},~~\exists f \in F([0,1])$ that $f(0) = x,$\\
    Thus $\phi$ is onto.
\end{revision}
%%%%%%%%%%%%%%
\begin{problem}[UD: 15.1]
    Find the compositions $f \circ g$ and $g \circ f$ assuming the domain of each is 
  the largest set of real numbers for which the function $f$, $g$, $f \circ g$ and 
  $g \circ f$ make sense. In your solution to each of the following, give the compositions 
  and the corresponding domain and range:\\
  (a) $f(x) = 1/(1 + x)$, $g(x) = x^{2}$;\\
  (b) $f(x) = x^{2}$, $g(X) = \sqrt{x}$;\\
  (c) $f(x) = 1/x$, $g(x) = x^{2} + 1$;
\end{problem}

\begin{cause}
    未理解清楚语义。
\end{cause}

\begin{revision}
  (a) $(f \circ g)(x) = 1/(1 + x^{2})$, $dom(f \circ g) = \mathbb{R}$, 
  $ran(f \circ g) = (0,1]$.\\
  $~~~~~~(g \circ f)(x) = \frac{1}{(1 + x)^{2}}$, $dom(g \circ f) = \mathbb{R} \backslash \{-1\}$, 
  $ran(g \circ f) = (0,+\infty)$.\\
  (b) $(f \circ g)(x) = x$, $dom(f \circ g) = [0,+\infty)$, $ran(f \circ g) = [0,+\infty)$.\\
  $~~~~~~(g \circ f)(x) = |x|$, $dom(g \circ f) = \mathbb{R}$, $ran(g \circ f) = [0,+\infty)$.\\
  (c) $(f \circ g)(x) = 1/(1 + x^{2})$, $dom(f \circ g) = \mathbb{R}$, 
  $ran(f \circ g) = (0,1]$.\\
  $~~~~~~(g \circ f)(x) = x^{-2} + 1$, $dom(g \circ f) = \mathbb{R} \backslash \{0\}$, 
  $ran(g \circ f) = (1,+\infty)$.
\end{revision}
%%%%%%%%%%%%%%%
\begin{problem}[UD: 16.22]
    Let $X$ be a nonempty set and let $A_{1}$ and $A_{2}$ be subsets of $X$. 
    Recall the notation for characteristic function, $\chi _{A}$, defined in 
    Problem 13.5.\\
    (a) If $\chi_{A_{1}} = \chi_{A_{2}}$, must $A_{1} = A_{2}$?\\
    (d) Can you find a similar result for $\chi_{X\backslash A_{1}}$?
\end{problem}

\begin{revision}
    (a) Yep.\\
    $~~~~~~\forall x \in X$, if $\chi_{A_{1}}(x) = \chi_{A_{2}}(x) = 1$,\\
    $~~~~~~$Then $x \in A_{1},~~x \in A_{2}$,\\
    $~~~~~~$If $\chi_{A_{1}}(x) = \chi_{A_{2}}(x) = 0$,\\
    $~~~~~~$Then $x \notin A_{1},~~x \notin A_{2}$,\\
    $~~~~~~$Thus $\forall x \in X$, if $x \in A_{1}$ then $x \in A_{2}$, 
    if $x \notin A_{1}$ then $x \notin A_{2}$,\\
    $~~~~~~$Thus $A_{1} = A_{2}$.\\
    (d) $-|\chi_{X \backslash A_{1}} - \chi_{X \backslash A_{2}}| + \chi_{A_{1} \cup A_{2}} 
    = \chi_{A_{1} \cap A_{2}}$
\end{revision}
%%%%%%%%%%%%%%%%%
\begin{problem}[UD: 22.1]
    Give an example, if possible, of each of the following:\\
  (b) a countably infinite collection of nonempty sets whose union is 
  finite;\\
  (c) a countably infinite collection of pairwise disjoint nonempty 
  sets whose union is finite.
\end{problem}

\begin{solution}
    (b) Impossible. Suppose $|A| = n$, then $|2_{A}| = 2_{n}$,\\
    $~~~~~~$If $A$ is an infinite collection of sets, then the union of the 
    sets is infinite too.\\
    (c) Impossible. Cuz if the union of sets in a collection is finite, then 
    the number of sets in the collection is finite too.
\end{solution}
%%%%%%%%%%%%%%
\begin{problem}[UD: 22.3]
    Is the set of all infinite sequences of 0's and 1's finite, countably 
  infinite, or uncountable? Guess and then prove, please.
\end{problem}

\begin{revision}
    Uncountable.\\
    Suppose it is countable, then such sequences can be listed as $a_{1}a_{2}a_{3}\cdots$, 
    $b_{1}b_{2}b_{3}\cdots$, $c_{1}c_{2}c_{3}\cdots$...\\
    Then we can find a sequence $p_{1}p_{2}p_{3}\cdots$ that 
    $a_{1} \neq p_{1},~b_{2} \neq p_{2},~c_{3} \neq p_{3}\cdots$\\
    We can easily see that $p_{1}p_{2}p_{3}\cdots$ is not in the collection 
    of $a_{1}a_{2}a_{3}\cdots$, $b_{1}b_{2}b_{3}\cdots$, $c_{1}c_{2}c_{3}\cdots$...\\
    Which means that the set is uncountable.
\end{revision}
%%%%%%%%%%%%%%%%%%%%
%\beginfb	% begin the feedback section (Optional)

%你可以写:
%\begin{itemize}
%  \item 对课程及教师的建议与意见
%  \item 教材中不理解的内容
%  \item 希望深入了解的内容
%  \item 等
%\end{itemize}
%%%%%%%%%%%%%%%%%%%%
\end{document}